% Options for packages loaded elsewhere
\PassOptionsToPackage{unicode}{hyperref}
\PassOptionsToPackage{hyphens}{url}
\PassOptionsToPackage{dvipsnames,svgnames,x11names}{xcolor}
%
\documentclass[
          a4paper,
        ]{article}
\raggedbottom
\usepackage{amsmath,amssymb}
\usepackage{setspace}
\usepackage{iftex}
\RequireLuaTeX

\usepackage{lscape} % für Querformat-Seiten

\usepackage{unicode-math} % this also loads fontspec
\defaultfontfeatures{Scale=MatchLowercase}
\defaultfontfeatures[\rmfamily]{Ligatures=TeX,Scale=1}

  \usepackage[]{plex-otf}
  
  
  
  
  
  
  
\directlua{luaotfload.add_fallback
   ("fallbacks",
    {
      "STIX Two Math:;",
      "NotoColorEmoji:mode=harf;",
    }
   )}
\defaultfontfeatures{RawFeature={fallback=fallbacks}}

\usepackage{academicons}
\usepackage{fontawesome5}
\usepackage{ccicons}

  
\IfFileExists{upquote.sty}{\usepackage{upquote}}{}
\IfFileExists{microtype.sty}{% use microtype if available
  \usepackage[]{microtype}
  \UseMicrotypeSet[protrusion]{basicmath} % disable protrusion for tt fonts
}{}
\usepackage{lua-widow-control}
  \makeatletter
\@ifundefined{KOMAClassName}{% if non-KOMA class
  \IfFileExists{parskip.sty}{%
    \usepackage{parskip}
  }{% else
    \setlength{\parindent}{0pt}
    \setlength{\parskip}{6pt plus 2pt minus 1pt}}
}{% if KOMA class
  \KOMAoptions{parskip=half}}
\makeatother
  
  
\usepackage[dvipsnames,svgnames,x11names]{xcolor}

\definecolor{oa-orange}     {RGB} {246, 130, 18}
\definecolor{ft-red}        {RGB} {168, 0, 0}
\definecolor{orcid-green}   {RGB} {166, 206, 57}
\definecolor{fortext-green}   {RGB} {26, 73, 76}


\colorlet{oa-orange}        {black}

\colorlet{highlightcolor1}  {DarkOrange}
\colorlet{highlightcolor2}  {DarkBlue}
\colorlet{highlightcolor3}  {DarkGreen}
\colorlet{highlightcolor4}  {DarkMagenta}

\colorlet{filecolor}        {highlightcolor1}
\colorlet{linkcolor}        {fortext-green}
\colorlet{citecolor}        {fortext-green}
\colorlet{urlcolor}         {fortext-green}


\NewDocumentCommand \oalogo { }
{%
  \textcolor{oa-orange}{\sffamily%\bfseries%
  open \aiOpenAccess\ access}
}


\NewDocumentCommand \orcidlink { m }
{%
	\texorpdfstring
	{\href{https://orcid.org/#1}{\textcolor{orcid-green}{\raisebox{-.2ex}{\aiOrcid}}}}
	{https://orcid.org/#1}%
}

\NewDocumentCommand \emaillink { m }
{%
	\texorpdfstring
	{\href{mailto:#1}{\textcolor{black}{\raisebox{-.2ex}{\faEnvelopeOpen[regular]}}}}
	{mailto:#1}%
}

  \usepackage{geometry}
\geometry{
	paper=a4paper,
	top=25mm,
	bottom=20mm,
	right=30mm,
	left=30mm,
	footskip=10mm,
	% showframe,
}
  
  \usepackage{listingsutf8}
\lstset{
  language=python,
  basicstyle=\ttfamily\footnotesize,
  columns=fullflexible,
  xleftmargin=2em,
  xrightmargin=2em,
  % frame=single,
  breaklines=true,
  postbreak=\mbox{\textcolor{highlightcolor1}{\(\hookrightarrow\)}\space},
}
\newcommand{\passthrough}[1]{#1}
\lstset{defaultdialect=[5.3]Lua}
\lstset{defaultdialect=[x86masm]Assembler}
  
  
  
\usepackage{longtable,booktabs,array}
\usepackage{tabularray}
\usepackage{ifthen}

  
\usepackage{calc} % for calculating minipage widths
% Correct order of tables after \paragraph or \subparagraph
\usepackage{etoolbox}
\makeatletter
\patchcmd\longtable{\par}{\if@noskipsec\mbox{}\fi\par}{}{}
\makeatother
% Allow footnotes in longtable head/foot
\IfFileExists{footnotehyper.sty}{\usepackage{footnotehyper}}{\usepackage{footnote}}
\makesavenoteenv{longtable}
\usepackage{graphicx}
\usepackage[export]{adjustbox}
\makeatletter
\def\maxwidth{\ifdim\Gin@nat@width>\linewidth\linewidth\else\Gin@nat@width\fi}
\def\maxheight{\ifdim\Gin@nat@height>\textheight\textheight\else\Gin@nat@height\fi}
\makeatother
\setkeys{Gin}{width=.8\maxwidth,height=\maxheight,keepaspectratio}
\makeatletter
\def\fps@figure{hbp}
\makeatother

  
\usepackage{luacolor}
\usepackage[soul]{lua-ul}

\setlength{\emergencystretch}{3em} % prevent overfull lines
\providecommand{\tightlist}{%
  \setlength{\itemsep}{0pt}\setlength{\parskip}{0pt}}

  \setcounter{secnumdepth}{-\maxdimen} % remove section numbering
  
  
  
\usepackage{fancyhdr}

\fancypagestyle{plain}{% used for the first page
	\fancyhf{}% clear all header and footer fields
	\fancyhead[L]{}
	\fancyfoot[L]{}
	\fancyfoot[R]{\small\thepage}
	\renewcommand{\headrulewidth}{0pt}%
	\renewcommand{\footrulewidth}{0pt}%
}

\fancypagestyle{page}{%
	\fancyhf{}% clear all header and footer fields
	\fancyhead[L]{}
	\fancyhead[R]{\small hefte, \textit{Selbststudieneinheit: Textannotation
(mit CATMA)}}
	\fancyfoot[L]{\small\textit{forTEXT} 2(13): Textannotation in der
Hochschullehre}
	\fancyfoot[R]{\thepage}
	\renewcommand{\headrulewidth}{0pt}%
	\renewcommand{\footrulewidth}{0pt}%
}

\pagestyle{page}


\usepackage[
	small,
	sf,bf,
	raggedright,
	clearempty,
]{titlesec}

\titleformat{\section}{\Large\sffamily\bfseries}{\thesection.}{.5em}{}
\titleformat{\subsection}{\large\sffamily\bfseries}{\thesubsection}{.5em}{}
\titleformat{\subsubsection}{\normalsize\sffamily\bfseries}{\thesubsubsection}{.5em}{}
\titleformat{\paragraph}{\normalsize\sffamily\bfseries}{\thesubsubsection}{.5em}{}
\titleformat{\subparagraph}{\normalsize\sffamily\bfseries}{\thesubsubsection}{.5em}{}

  \NewDocumentCommand\citeproctext{}{}
\NewDocumentCommand\citeproc{mm}{%
  \begingroup\def\citeproctext{#2}\cite{#1}\endgroup}
\makeatletter
 \let\@cite@ofmt\@firstofone
 \def\@biblabel#1{}
 \def\@cite#1#2{{#1\if@tempswa , #2\fi}}
\makeatother
\newlength{\cslhangindent}
\setlength{\cslhangindent}{1.5em}
\newlength{\csllabelwidth}
\setlength{\csllabelwidth}{3em}
\newenvironment{CSLReferences}[2] % #1 hanging-indent, #2 entry-spacing
 {\begin{list}{}{%
  \setlength{\itemindent}{0pt}
  \setlength{\leftmargin}{0pt}
  \setlength{\parsep}{0pt}
  % turn on hanging indent if param 1 is 1
  \ifodd #1
   \setlength{\leftmargin}{\cslhangindent}
   \setlength{\itemindent}{-1\cslhangindent}
  \fi
  % set entry spacing
  \setlength{\itemsep}{#2\baselineskip}}}
 {\end{list}}
\usepackage{calc}
\newcommand{\CSLBlock}[1]{\hfill\break\parbox[t]{\linewidth}{\strut\ignorespaces#1\strut}}
\newcommand{\CSLLeftMargin}[1]{\parbox[t]{\csllabelwidth}{\strut#1\strut}}
\newcommand{\CSLRightInline}[1]{\parbox[t]{\linewidth - \csllabelwidth}{\strut#1\strut}}
\newcommand{\CSLIndent}[1]{\hspace{\cslhangindent}#1}
  
  \usepackage[bidi=basic]{babel}
    \babelprovide[main,import]{ngerman}
                  \let\LanguageShortHands\languageshorthands
\def\languageshorthands#1{}
  
\usepackage[
	ragged,
	bottom,
	norule,
	multiple,
]{footmisc}

\makeatletter
\RenewDocumentCommand \footnotemargin { } {0em}
\RenewDocumentCommand \thefootnote { } {\arabic{footnote}}
\RenewDocumentCommand \@makefntext { m } {\noindent{\@thefnmark}. #1}
\interfootnotelinepenalty=10000
\makeatother

\usepackage{changepage}
% \newlength{\overhang}
% \setlength{\overhang}{\marginparwidth}
% \addtolength{\overhang}{\marginparsep}

  \usepackage{caption}
\captionsetup{labelformat=empty,font={small,it}}
  
\usepackage{selnolig}

  
  
  
  
\usepackage{csquotes}
\usepackage{bookmark}
\IfFileExists{xurl.sty}{\usepackage{xurl}}{} % add URL line breaks if available
\urlstyle{same}

  
  

\makeatletter
\def\@maketitle{%
	%
	% title
	%
	\newpage
	\null
	\vspace*{-\topskip}
	\begin{tblr}{
    vline{1,5} = {1pt},
    colspec={lllX[c,l]},
    width=\textwidth,
    columns={font=\small\sffamily,},
    column{1,3}={
      rightsep=.3em,
      font=\footnotesize\sffamily,
    },
}
\hline[1pt]
\SetCell[c=4]{t,l}{\normalsize\textbf{Selbststudieneinheit:
Textannotation (mit CATMA)}} & & & \\
& & & \SetCell[r=3]{c,r}{\includegraphics[height=2\baselineskip]{}} \\
\SetCell[c=3]{t,l}{\small fortext hefte \emaillink{fortext-hefte.de}
\textsuperscript{\scriptsize 1}
} & & &\\
\SetCell[c=2]{t,l}{% 
\footnotesize%
2. FH Darmstadt \\
} & & & \\
		\hline[1pt]
    Thema:                & Textannotation in der Hochschullehre & & \\
    DOI:                  & \href{https://doi.org/}{10.48694/abc} & & \\
    Jahrgang:             & 2 & & \\
    Ausgabe:              & 13 & & \\
		Erscheinungsdatum:    &  & & \\
		Lizenz:               & \faCreativeCommons\ \faCreativeCommonsBy\ \faCreativeCommonsSa\ & & \oalogo  \\
		\hline[1pt]
	\end{tblr}
}
\makeatother

\hypersetup{
      pdftitle={Selbststudieneinheit: Textannotation (mit CATMA)},
      pdfauthor={fortext hefte},
      pdflang={de},
      pdfsubject={forTEXT 2(13): Textannotation in der Hochschullehre},
      pdfkeywords={keyword1, keyword2, keyword3, keyword4, keyword5},
      colorlinks=true,
  linkcolor={linkcolor},
  filecolor={filecolor},
  citecolor={citecolor},
  urlcolor={urlcolor},
  pdfcreator={LuaLaTeX via pandoc}
}


  
  \usepackage{ragged2e}
\usepackage[section]{placeins}
% Manage float placement
\usepackage{float}
\floatplacement{figure}{H}

\usepackage{marginnote}
\RenewDocumentCommand \marginfont { }
{ \sffamily\footnotesize }
\RenewDocumentCommand \raggedleftmarginnote { } { }
% \author{true}

% \date{}

\begin{document}


\pagestyle{plain}


\maketitle



% \marginnote{\RaggedRight}[30\baselineskip]%









\setstretch{1.2}


\pagestyle{page}

\renewcommand{\arraystretch}{2}  % Zeilenabstand in Tabellen erhöhen

\section{Inhaltsverzeichnis}\label{inhaltsverzeichnis}

\begin{enumerate}
\def\labelenumi{\arabic{enumi}.}
\tightlist
\item
  \hyperref[einfuxfchrung]{Einführung}
\item
  \hyperref[gesamtablauf]{Gesamtablauf}
\item
  \hyperref[sitzungen-im-detail]{Sitzungen im Detail}
\item
  \hyperref[reflexion]{Reflexion}
\end{enumerate}

\section{Einführung}\label{einfuxfchrung}

Das vorliegende Lehrkonzept stellt eine vierteilige Selbststudieneinheit
zum Thema „Textannotation (mit CATMA)`` vor. Es handelt sich um eine
Kombination aus asynchroner und synchroner Lehre, die über einen
Zeitraum von vier Semesterwochen im Rahmen der Einführungsveranstaltung
„Grundkurs Literaturwissenschaft 2`` im Sommersemester an der
Technischen Universität Darmstadt durchgeführt wird. Der Grundkurs 2 ist
der zweite Teil des zweiteiligen Einführungsmoduls zur Einführung in die
Literaturwissenschaft, das im Germanistikstudium in den ersten zwei
Studiensemestern die Grundlagen der deutschsprachigen
Literaturwissenschaft abdecken soll, um die Studierenden auf das
literaturwissenschaftliche Arbeiten in den darauf aufbauenden Pro- und
Hauptseminaren der höheren Semester vorzubereiten. Die
Selbststudieneinheit repräsentiert eine Schnittstelle zwischen
Geisteswissenschaften und Digital Humanities. Sie integriert digitale
Methoden in die literaturwissenschaftliche Forschung und Lehre, indem
sie den Studierenden die Nutzung von CATMA (Computer Assisted Text
Markup and Analysis) als Analysetool nahebringt. Diese
Interdisziplinarität ermöglicht es, traditionelle
geisteswissenschaftliche Fragestellungen durch den Einsatz von
Technologien zu vertiefen und zu erweitern, und fördert gleichzeitig die
digitalen Kompetenzen der Studierenden. Die Selbststudieneinheit deckt
zeitlich 1/5 der Semesterwochen ab und ist vor allem im Sommersemester
zur thematischen Abdeckung der Wochen mit Feiertagsausfällen synchroner
Sitzungen praktisch im Semesterplan zu integrieren. Geprüft wird der
erfolgreiche Abschluss der Lehrveranstaltung durch eine 90-minütige
schriftliche Klausur am Ende der Vorlesungszeit. Nach erfolgreichem
Bestehen der Abschlussklausur erhalten die Teilnehmenden 5 ECTS-Punkte.
In der Abschlussklausur ist eine Frage zur Textannotation vorgesehen, um
das Erreichen der Lernziele der Selbststudieneinheit zu überprüfen. Der
Grundkurs 2 und die Selbststudieneinheit wurden in dieser Form bereits
dreimal durchgeführt: im Sommersemester 2022, 2023 und 2024. Der Kurs
wird als synchrone Lehrveranstaltung in Lehrform eines Grundkurses mit
zwei Semesterwochenstunden für eine Gruppe von ca. 20-30 Studierenden
(Germanistik Bachelor/ Deutsch Lehramt an Gymnasien) angeboten, die
i.d.R. aber nicht zwingend bereits den ersten Teil des Grundkurses
besucht haben und somit erste grundlegende Erfahrungen in
Literaturwissenschaft mitbringen. Die vorgesehenen Lerninhalte des
Grundkurses als Rahmenlehrveranstaltung sind laut Modulbeschreibung die
„Einführung in erweiterte Gebiete der Literaturwissenschaft. Studierende
sollen am Ende des Kurses mit Themen der Narrationstheorie, der
Literaturgeschichte und der Editionswissenschaft sowie mit den
entsprechenden Theorien und Konzepten vertraut sein und diese unter
Anleitung kritisch einordnen und diskutieren können''. Die unter den
Literaturwissenschaftsdozent*innen abgestimmten großen Themen der
Lehrveranstaltung Grundkurs Literaturwissenschaft 2 umfassen: zwei
Sitzungen zu Literaturtheorien, zwei Sitzungen zur Literaturgeschichte
(19. und 20. Jahrhundert), vier Sitzungen zur Großgattung Prosa, drei
Sitzungen zur Großgattung Lyrik sowie drei Sitzungen zu
Organisatorischem, Klausurvorbereitung und Klausurdurchführung. Eine
erfolgreiche Teilnahme am Seminar befähigt die Studierenden zum Umgang
mit Begriffen und Konzepten erweiterter Gebiete der
Literaturwissenschaft. Sie können Analysen mittels wichtiger Methoden
des jeweiligen Teilgebiets durchführen. Darüber hinaus haben sie ein
grundlegendes Verständnis der Literaturwissenschaft und ihrer
Unterdisziplinen erlangt und sind mit den Grundlagen der
literaturwissenschaftlichen Analyse, dem analytischen Lesen und dem
wissenschaftlichen Arbeiten vertraut. Zur Vermittlung von Kompetenzen
wurden verschiedene Medien eingesetzt. In der asynchronen
Selbststudieneinheit haben die Studierenden Zugang zu Videotutorials.
Diese Tutorials bieten eine Einführung in die manuelle Annotation mit
dem digitalen Tool CATMA, sowohl für Anfänger als auch für
Fortgeschrittene. Zudem wurden vorbereitend drei Einführungstexte zum
manuellen und kollaborativen Annotieren (analog und digital) zur
Verfügung gestellt. Während der synchronen Lehrveranstaltungen wurden
Primärtexte, insbesondere die Erzählung „Krambambuli`` von Marie von
Ebner-Eschenbach, diskutiert. Zur Unterstützung der Studierenden wurde
ein*e Tutor*in eingesetzt. Diese*r stand während der asynchronen
Selbststudieneinheiten über Online-Plattformen für Fragen und
Hilfestellungen zur Verfügung. Während der synchronen Sitzungen halfen
er/sie bei technischen Problemen mit dem Annotationstool CATMA und
unterstützten die Studierenden bei der Anwendung der
Annotationsmethoden. Als Ausstattung zur Durchführung der
Selbststudieneinheit müssen die Studierenden Zugang zu einem
internetverbundenen Laptop haben und einen der gängigen Browser
verwenden können (z.B. Firefox, Chrome oder Safari) sowie grundlegende
Sprachkenntnisse im Englischen mitbringen, um das Optical User Interface
der Textannotationssoftware CATMA verstehen und nutzen zu können. Für
die synchronen Veranstaltungssitzungen benötigt die/der Lehrende einen
internetverbundenen Laptop sowie einen Beamer.

\begin{landscape}

\begin{longtable}[]{@{}
  >{\raggedright\arraybackslash}p{(\columnwidth - 14\tabcolsep) * \real{0.0050}}
  >{\raggedright\arraybackslash}p{(\columnwidth - 14\tabcolsep) * \real{0.1000}}
  >{\raggedright\arraybackslash}p{(\columnwidth - 14\tabcolsep) * \real{0.1500}}
  >{\raggedright\arraybackslash}p{(\columnwidth - 14\tabcolsep) * \real{0.2000}}
  >{\raggedright\arraybackslash}p{(\columnwidth - 14\tabcolsep) * \real{0.2000}}
  >{\raggedright\arraybackslash}p{(\columnwidth - 14\tabcolsep) * \real{0.1500}}
  >{\raggedright\arraybackslash}p{(\columnwidth - 14\tabcolsep) * \real{0.1000}}
  >{\raggedright\arraybackslash}p{(\columnwidth - 14\tabcolsep) * \real{0.1000}}@{}}
\toprule\noalign{}
\begin{minipage}[b]{\linewidth}\raggedright
Einheit
\end{minipage} & \begin{minipage}[b]{\linewidth}\raggedright
Modus
\end{minipage} & \begin{minipage}[b]{\linewidth}\raggedright
Thema
\end{minipage} & \begin{minipage}[b]{\linewidth}\raggedright
Inhalt
\end{minipage} & \begin{minipage}[b]{\linewidth}\raggedright
Lernziel
\end{minipage} & \begin{minipage}[b]{\linewidth}\raggedright
Vorbereitung
\end{minipage} & \begin{minipage}[b]{\linewidth}\raggedright
Für Lehrende
\end{minipage} & \begin{minipage}[b]{\linewidth}\raggedright
Abgabe/ Aufgabe
\end{minipage} \\
\midrule\noalign{}
\endhead
\bottomrule\noalign{}
\endlastfoot
1 & Synchron präsenz & Theoretische Einführung in die Analyse von
Figuren & Einführung in die Figurenanalyse und das Formulieren von
Forschungsfragen & Grundlagen der Figurenanalyse verstehen und anwenden;
Forschungsfragen formulieren & Lektüre
(\citeproc{ref-lahnFiguren2016}{Lahn und Meister 2016}) & Laptop/
Beamer/ vorbereitende Texte & Formulieren einer Forschungsfrage zur
Figurenanalyse \\
2 & Asynchron online & Selbststudium: Einführung in die
literaturwissenschaftliche Textannotation & Erarbeitung von
Einführungstexten und Video-Tutorials zur manuellen und digitalen
Annotation & Grundlagen des manuellen und digitalen Annotierens
erlernen; Anwendung des Tools CATMA & Lesen von
(\citeproc{ref-horstmannLerneinheitManuelleAnnotation2019}{Horstmann
2024}) Anschauen der Tutorials
(\citeproc{ref-fortextTutorialCATMAAnnotieren2020}{forTEXT 2020}) &
Sicherstellen dass Materialien verfügbar sind & Abschluss der
Materialien/ Vorbereitung auf die synchrone Sitzung \\
3 & Synchron präsenz & Einführung in die literaturwissenschaftliche
Annotationspraxis & Praktische Anwendung der Annotationsmethoden mit
CATMA & Anwendung der Annotationsmethoden; Reflexion der Nützlichkeit
der Annotation & Vorbereitung der Annotationsbeispiele & Laptop/ Beamer/
Zugriff auf CATMA & Eigenständige Annotation einer Textpassage \\
4 & Synchron präsenz & Überprüfung des Gelernten durch Anwendungsaufgabe
& Diskussion und Übung von Prüfungsfragen zur Textannotation &
Sicherstellung der Lernzielerreichung; Vorbereitung auf die Klausur &
Anwendungsaufgabe vorbereiten & Bereitstellung der Aufgabe & Formulieren
einer Forschungsfrage/ Begründung der Relevanz/ Anwendung von
Textannotation zur Beantwortung der Forschungsfrage \\
& & & & & & & \\
\end{longtable}

\end{landscape}

\section{Gesamtablauf}\label{gesamtablauf}

Die Selbststudieneinheit verortet sich im Semesterplan als Teil des
Themas Prosa und baut auf den Inhalten der Sitzungen zu
Literaturgeschichte des 19. Jahrhunderts und Literaturtheorie auf. Sie
besteht aus vier Teilen: Der erste Teil beinhaltet eine theoretische
Einführung in die Analyse von Figuren und in das Formulieren von
literaturwissenschaftlichen Fragestellungen in einer synchronen
Lehrveranstaltungssitzung. Im zweiten Teil findet eine asynchrone
Lerneinheit über zwei Wochen statt, die praktisch als
Feiertagsüberbrückung dient. In dieser Phase erarbeiten die Studierenden
zusammengestellte Materialien zur Einarbeitung in die
literaturwissenschaftliche Textannotation. Diese Materialien umfassen
drei Einführungstexte ins manuelle und kollaborative Annotieren analog
und digital sowie zwei Video-Tutorials (beginner und advanced) zur
Einführung in die manuelle Annotation mit dem digitalen Annotationstool
CATMA. Diese Materialien stammen von der an der TU Darmstadt
beheimateten Forschungsgruppe forTEXT Literatur digital erforschen. Der
dritte Teil besteht aus einer synchronen Lehrveranstaltung zur
Einführung in die literaturwissenschaftliche Annotationspraxis. Im
vierten und letzten Teil wird das Gelernte durch eine Prüfungsfrage in
der Klausur überprüft, die sich auf die literaturwissenschaftliche
Textannotation fokussiert. Die Selbststudieneinheit zielt darauf ab, die
Lerninhalte Narrationstheorie und die Großgattung Prosa, die Erzählung
„Krambambuli`` von Marie von Ebner-Eschenbach sowie die Einführung und
Anwendung der literaturwissenschaftlichen Methode des Annotierens,
sowohl manuell analog als auch manuell digital mit dem Annotationstool
CATMA zu vermitteln. Zudem beinhaltet die Einheit die Einzeltextanalyse
der Primärlektüre mit einem besonderen Fokus auf die
Figurencharakterisierung. Die Qualifikationsziele und Lernergebnisse
umfassen die Vertiefung der Kenntnisse zur Narrationstheorie und ihrer
Methodenanwendung. Die Studierenden sollen in der Lage sein, Methoden
der Narrationstheorie auf bekannte literarische Primärtexte des 19.
Jahrhunderts anzuwenden und drei literaturwissenschaftliche
Annotationsmethoden kennenzulernen: manuell-analoges Annotieren
literarischer Texte, manuell-digitales Annotieren literarischer Texte
und kollaboratives manuell-digitales Annotieren literarischer Texte. Des
Weiteren sollen die Studierenden die Nützlichkeit
literaturwissenschaftlicher Textannotation reflektieren und lernen,
situationsabhängig zu entscheiden, wie diese Methode am sinnvollsten
anzuwenden ist, um sich einer ausgewählten Forschungsfrage zu nähern.
Schließlich sollen sie literaturwissenschaftliche Forschungsfragen
formulieren und sich der Beantwortung dieser Fragen mit Hilfe
literaturwissenschaftlicher Textannotation annähern können.

\section{Sitzungen im Detail}\label{sitzungen-im-detail}

\subsection{Sitzung 1: Theoretische Einführung in die Analyse von
Figuren}\label{sitzung-1-theoretische-einfuxfchrung-in-die-analyse-von-figuren}

In der ersten Sitzung erhalten die Studierenden eine theoretische
Einführung in die Analyse von Figuren
(\citeproc{ref-lahnFiguren2016}{Lahn und Meister 2016}) sowie in das
Formulieren von literaturwissenschaftlichen Fragestellungen. Diese
synchrone Lehrveranstaltung beginnt mit einem Vortrag, der die
grundlegenden Konzepte und Methoden der Figurenanalyse behandelt. Dabei
wird die Bedeutung und Funktion von Figuren in literarischen Texten
erläutert. Verschiedene Ansätze der Figurenanalyse werden vorgestellt
und diskutiert. Im Anschluss an den theoretischen Input wird das
Formulieren von literaturwissenschaftlichen Forschungsfragen
thematisiert. Die Studierenden lernen, wie sie präzise und relevante
Fragen zur Figurenanalyse entwickeln können. Die Sitzung umfasst
Diskussionen und Gruppenarbeit, bei der die Studierenden anhand eines
kurzen Textausschnitts erste Analysen durchführen und Forschungsfragen
formulieren. Ziel dieser Sitzung ist es, ein Verständnis der
grundlegenden Konzepte der Figurenanalyse zu vermitteln, die Fähigkeit
zur Unterscheidung und Anwendung verschiedener Ansätze der
Figurenanalyse zu entwickeln und die Kompetenz zu fördern,
literaturwissenschaftliche Forschungsfragen zu formulieren.

\subsection{Sitzung 2: Asynchrone Selbststudieneinheit über zwei
Wochen}\label{sitzung-2-asynchrone-selbststudieneinheit-uxfcber-zwei-wochen}

Die zweite Sitzung besteht aus einer asynchronen Selbststudieneinheit,
welche sich über zwei Wochen erstreckt. Der Abschnitt bietet sich daher
beispielsweise als Feiertagsüberbrückung an. In dieser Zeit haben die
Studierenden die Gelegenheit, sich eigenständig in die Grundlagen der
literaturwissenschaftlichen Textannotation einzuarbeiten. Sie setzen
sich mit drei Einführungstexten auseinander, die das manuelle und
kollaborative Annotieren sowohl in analoger als auch digitaler Form
behandeln
(\citeproc{ref-horstmannLerneinheitManuelleAnnotation2019}{Horstmann
2024}). Ergänzend dazu stehen zwei Video-Tutorials
(\citeproc{ref-fortextTutorialCATMAAnnotieren2020}{forTEXT 2020}) zur
Einführung in die manuelle Annotation mit dem digitalen Annotationstool
CATMA zur Verfügung. Diese Materialien bieten den Studierenden eine
umfassende Einführung in die Annotationsmethoden und ermöglichen ihnen,
das Gelernte praktisch umzusetzen. Ziel dieser Einheit ist es, die
Grundlagen des manuellen und digitalen Annotierens zu vermitteln, die
Fähigkeit zur Nutzung des Tools CATMA zu entwickeln und ein Verständnis
für die Vor- und Nachteile verschiedener Annotationsmethoden zu
schaffen. Die Studierenden lesen die Einführungstexte, schauen sich die
Video-Tutorials an und führen praktische Übungen zur Annotation mit
CATMA durch.

\subsection{Sitzung 3: Synchrone Lehrveranstaltung zur Einführung in die
literaturwissenschaftliche
Annotationspraxis}\label{sitzung-3-synchrone-lehrveranstaltung-zur-einfuxfchrung-in-die-literaturwissenschaftliche-annotationspraxis}

In der dritten Sitzung steht die praktische Anwendung der
Annotationsmethoden im Fokus. Die Studierenden bringen die während der
asynchronen Selbststudieneinheit erworbenen Kenntnisse mit und setzen
diese in einer synchronen Lehrveranstaltung in die Praxis um. Nach einer
kurzen Einführung und Wiederholung der wichtigsten Punkte aus der
Selbststudieneinheit führen die Studierenden unter Anleitung eigene
Annotationen durch. Es werden konkrete Annotationsaufgaben gestellt, die
mit CATMA gelöst werden sollen. Die Studierenden arbeiten dabei zunächst
individuell. Im Anschluss werden im Plenum Ergebnisse, Schwierigkeiten
und Fragen diskutiert. Ziel dieser Sitzung ist es, die Anwendung der
Annotationsmethoden in der Praxis zu vertiefen und die Fähigkeit zur
kritischen Reflexion von Textannotation zu fördern.

\subsection{Sitzung 4: Überprüfung des Gelernten durch eine
Anwendungsaufgabe}\label{sitzung-4-uxfcberpruxfcfung-des-gelernten-durch-eine-anwendungsaufgabe}

In der vierten Sitzung wird das in den vorangegangenen Einheiten
Gelernte anhand einer praktischen Aufgabe (siehe Aufgabe 1) angewendet.
Die Studierenden erhalten einen Primärtextabschnitt, auf dessen
Grundlage sie eine präzise und relevante Forschungsfrage formulieren
müssen. Diese Frage sollte sich auf die im Kurs behandelten Themen wie
Narrationstheorie, Figurenanalyse oder andere relevante Aspekte
beziehen. Zusätzlich sollen die Studierenden die
literaturwissenschaftliche Relevanz ihrer Forschungsfrage erläutern,
indem sie diese in den Forschungskontext einordnen. Anschließend müssen
sie entscheiden, ob und wie die Methode der Textannotation verwendet
werden kann, um ihre Fragestellung zu beantworten. Dies umfasst die
Begründung der Entscheidung und eine Beschreibung der möglichen
Vorgehensweise. Ziel dieser Sitzung ist die Entwicklung einer
literaturwissenschaftlichen Fragestellung, die durch die Methode der
Textannotation bearbeitet werden kann. Anhand der Aufgabe zeigen die
Studierenden die Fähigkeit, theoretische Konzepte auf konkrete Analysen
anzuwenden.

\subsubsection{Aufgabe 1:}\label{aufgabe-1}

Lesen Sie sich den \emph{Primärtextabschnitt XY} von
\emph{Musterautor*in} durch (Primärtext nach Wahl). Formulieren Sie eine
Forschungsfrage hinsichtlich der im Kurs behandelten Inhalte (zu
Prosa/Narrationstheorie: Diskurs, Geschichte, Figuren,
Literaturgeschichte oder anderem. Beachten Sie bei der Formulierung die
im Kurs besprochenen inhaltlichen und formalen Anforderungen an eine
Forschungsfrage (1 Satz, 2 Punkte). Begründen Sie die
(literaturwissenschaftliche) Relevanz Ihrer Forschungsfrage kurz, indem
Sie sie in den literaturwissenschaftlichen Forschungskontext einordnen.
Entscheiden Sie, ob und wie literaturwissenschaftliche Textannotation
als Grundlage der Annäherung an die Beantwortung Ihrer Forschungsfrage
angewendet werden kann. Begründen Sie Ihre Entscheidung und beschreiben
Sie Ihre mögliche Herangehensweise.

\section{Reflexion}\label{reflexion}

Insgesamt bin ich zufrieden mit dem Verlauf der Selbststudieneinheit
sowie der aktiven Beteiligung der Studierenden. Meine Bedenken, dass die
geplanten Zeitabschnitte für Arbeitsaufgaben und Plenumsdiskussionen zu
eng getaktet sein könnten, hatten sich nicht bestätigt. Dies lag vor
allem daran, dass ich die Zeitplanung durchgängig im Blick hatte und
Diskussionen so moderierte, dass wir den Zeitrahmen einhielten.
Retrospektiv konnte ich feststellen, dass die Vierteilung des
Lehrprojekts eine gute Idee war und die Studierenden das Angebot einer
asynchronen Selbststudieneinheit anstelle einer synchronen Sitzung gut
annahmen. Den Erfolg der Selbststudieneinheit konnte ich überprüfen,
weil die Studierenden, die an der synchronen Sitzung teilnahmen, sie als
Vorbereitung durchgeführt haben mussten, um während der Sitzung gut
mitarbeiten zu können. Auch hatten fast alle Teilnehmenden einen Laptop
mit funktionierender Internetverbindung dabei, sodass sie die
Anwendungsaufgaben durchführen konnten. Beim Durchgehen während der
Bearbeitungszeit der Anwendungsaufgaben konnte ich außerdem überprüfen,
ob die grundsätzliche Verwendung von CATMA verstanden worden war. Ich
schaute auf die Bildschirme der Teilnehmenden und gab aktionales
Feedback zu den Arbeitsschritten, half bei Schwierigkeiten und
beantwortete individuelle Fragen. Des Weiteren passte es auch gut, dass
die Teilnehmenden das Gelernte aus der Selbststudieneinheit und den
Theorieinputvorträgen immer direkt selbst anwenden konnten und zum
Weiterdenken angeregt wurden, indem wir Ideen sammelten für mögliche
weitere Fragestellungen und Forschungsideen, die mit Textannotation
angegangen werden können. Einige Punkte, die ich bei einer erneuten
Durchführung des Lehrprojekts besser machen würde, beziehen sich auf die
Zeiteinteilung, einzelne Erweiterungen in der Anleitung der
Anwendungsaufgaben sowie auf mein Sprechtempo und die Wahl der
Textgrundlage. Zunächst würde ich nach der Unterrichtsphase 6 und vor
dem zweiten Inputvortrag eine Pause einfügen, damit die Studierenden für
den zweiten Teil der Veranstaltung mehr Energie und Aufmerksamkeit haben
sowie die gelernten Inhalte etwas sacken lassen können. Je nach
Gruppengröße sowie bei erwartbar mehr technischen Hürden bei den
Lernenden wäre auch eine Zweiteilung der asynchronen Sitzungen auf zwei
Sitzungen denkbar. Auch würde ich meine Anleitung der Anwendungsaufgabe
zur Annotation mit CATMA dahingehend erweitern, dass ich noch mehr
Screenshots zeige und Hinweise gebe, wie der Arbeitsbereich in dem
Annotationstool eingerichtet wird. Dies würde den Studierenden helfen,
sich schneller zurechtzufinden und effizienter arbeiten zu können.
Schließlich werde ich die Veranstaltung generell entzerren, indem ich
mir mehr Zeit für die Inputvorträge und Erklärungen lasse und mehr Zeit
für die Arbeitsphasen einplane. Zum Beispiel könnte ich es auf eine
dreistündige Veranstaltung ausdehnen, indem ich die Studierenden länger
selbst annotieren lasse und am Ende zusätzlich zu meinen vorbereiteten
Ergebnissen auch die Ergebnisse der Studierenden in einer Liveanalyse
per Beamerpräsentation aufbereite und interpretieren lasse. Dies würde
den Lernprozess vertiefen und den Studierenden ermöglichen, ihre
Annotationsfähigkeiten in einem realistischeren Rahmen zu üben.
Außerhalb des Grundkurskontextes könnte ich zudem einen kürzeren Text
zur Annotation zur Verfügung stellen, damit die Teilnehmenden einen
vollständigen Text annotieren können und nicht nur einen kurzen
Ausschnitt. Dafür könnte man zum Beispiel ein Märchen als Textgrundlage
wählen. Ein kürzerer Text würde es den Studierenden ermöglichen, den
gesamten Text zu erfassen und somit ein besseres Verständnis für die
Struktur und die narrativen Techniken zu entwickeln. Die Unterstützung
durch eine*n Tutor*in erwies sich als sehr hilfreich, insbesondere bei
technischen Fragen und individuellen Problemen. Der/Die Tutor*in konnten
direkt vor Ort oder online Hilfestellungen geben und die Studierenden
bei der Anwendung von CATMA unterstützen. In Bezug auf das Feedback der
Studierenden wurde deutlich, dass sie die Mischung aus synchronen und
asynchronen Einheiten sowie die praxisorientierte Anwendung der Theorie
schätzten. Zusammenfassend lässt sich sagen, dass das Lehrkonzept in
seiner derzeitigen Form weitgehend erfolgreich war, jedoch durch einige
Anpassungen in der Struktur und im Ablauf weiter optimiert werden kann.

\section*{Bibliographie}\label{bibliography}
\addcontentsline{toc}{section}{Bibliographie}

\phantomsection\label{refs}
\begin{CSLReferences}{1}{0}
\bibitem[\citeproctext]{ref-fortextTutorialCATMAAnnotieren2020}
forTEXT. 2020. Tutorial: In CATMA 6 annotieren. Manuelle Annotation und
Literaturanalyse. 20. Januar. doi:
\href{https://doi.org/10.5281/zenodo.10353910}{10.5281/zenodo.10353910},
\url{https://doi.org/10.5281/zenodo.10353910}.

\bibitem[\citeproctext]{ref-horstmannLerneinheitManuelleAnnotation2019}
Horstmann, Jan. 2024. Lerneinheit: Manuelle Annotation mit CATMA.
\emph{forTEXT Heft} 1, Nr. 4. Manuelle Annotation (30. Juli). doi:
\href{https://doi.org/10.48694/fortext.3750}{10.48694/fortext.3750},
\url{https://fortext.net/routinen/lerneinheiten/manuelle-annotation-mit-catma}.

\bibitem[\citeproctext]{ref-lahnFiguren2016}
Lahn, Silke und Jan Christoph Meister. 2016. Figuren. In:
\emph{Einführung in die Erzähltextanalyse}, 234--249. 3., aktualisierte
und erweiterte Auflage. Stuttgart: J.B. Metzler.

\end{CSLReferences}




\end{document}
